\subsection{Analisis Socioeconómico de Universall}

En el presente capítulo se trata de abordar el funcionamiento del ecosistema completo de Universall, desde la recogida de información por parte de los sensores hasta su posterior análisis mediante Big Data.

\textbf{ELEMENTOS QUE INTERVIENEN}
En primer lugar definimos la función de cada uno de los componentes de Universall:
\begin{itemize}
\item Quark: es el conjunto de sensores que capta la información del entorno
\item Omega: es la base de comunicación que recoge la información de los quarks y la transmite a los servidores para posteriormente proceder a su análisis.
\end{itemize}

\textbf{FUNCIONAMIENTO}
El funcionamiento de Universall se divide en varias etapas que describimos a continuación que se pueden ver en la FIGURA

\begin{itemize}
\item Recopilación de información:
\end{itemize}

\begin{itemize}
\item Conexión con Omega:
\end{itemize}

\begin{itemize}
\item Subida de información a los servidores:
\end{itemize}

\begin{itemize}
\item Análisis de los datos:
\end{itemize}



\textbf{RIESGOS}



