\subsection{Analisis Socioeconómico de Universall}

En este apartado se va a llevar  a cabo un analisis socioeconómico del ecosistema  Universall.

\textbf{ORIGEN}

Universall surge como una solución genérica para satiacer las necesidades de las empresas que quieran actulizarse para comenzar a estar conectados. Satisfacer dichas necesidades de una manera exclusiva en función del cliente provocaba el encarecimiento del producto destinado al cliente, así como su tiempo de entrega. Es este el factor clave del nacimiento de Universall, ya que es capaz de de gestionar las necesidades más demandadas y comunes, lo que hace que el tiempo de entrega sea mucho menor y su precio se vea claramente reducido pues el dispositivo es el mismo para cualquier cliente. 

\textbf{NECESIDADES}

Las necesidades que pueda llegar a satisfacer este producto van a estar siempre relacionadas con las exigencias del cliente y la apliación final que el mismo cliente busque. Es decir, no solo dependerá del hardware, sino tambien del sowtfare y del análisis que se realice de los datos recogidos. 

\textbf{APLICACIONES}

Como hemos comentado anteriormente, las aplicaciones siempre dependerán de las exigencias del cliente. Geeks!me ha desarrollado un cojunto de soluciones y casos de uso que incluyen tanto el hardware como el software para explotar el potencial de Universall, las cuales son:

\begin{itemize}
\item Universall Space™: destinada a hacer los espacios inteligentes y optimizar su uso. Con el se podría conocer datos como el tiempo medio de los descansos de un empleado o si una sala de reuiniones está libre o no en tiempo real.
\item Universall Care™: su objetivo principal es el cuidado de las personas mayores. Con el se podría hacer un seguimiento a distancia de la rutina que está llevando la persona, saber si ha salido de casa o si por ejemplo ha abierto el armario de las medicinas.
\item Universall Retail™: enfocado a los pequeños y grandes comercios con el que se podrán conocer el número de personas que entran al establecimiento o si por ejemplo la puerta está abierta o cerrada.
\item Universal Stars™: su función reside en el análisis del sueños de una persona. Esta apliación mostrará la actividad del sueño que esa persona ha tenido, y con la que se podrá conocer cómo descansa e incluso podrá aprender a dormir mejor.
\item Universal Safe™: esta aplicación está destinada a conocer el estado de nuestras propiedades en tiempo real, desde puertas y ventanas de la vivienda hasta un trastero.
\end{itemize}


\textbf{PRECIO}

El precio también varía en función de las necesidades del cliente. En términos generales se pueden establecer un rango de precios:
\begin{itemize}
\item Quarks: 25-30 €
\item Omega: 70-80€
\end{itemize}

\textbf{¿A QUIÉN VA DEDICADO?}

Universall busca satisfacer las demadas que puedan tener tanto una empresa como un particular, por lo que su modelo de negocio se puede resumir en dos vertientes: 

\begin{itemize}
\item B2B (Business to Business):
\item B2C (Business to Consumer):
\end{itemize}



\textbf{DAFO DE UNIVERSALL}

\begin{enumerate}
\item DEBILIDADES
\begin{itemize}

\item Es genérico. No está fabricado para un único cliente.
\item Cumple necesidades genéricas preestablecidas. No puede satisfacer otras necesidades fuera de ese rango.
\item Hardware y Software de código cerrado
\end{itemize}


\item FORTALEZAS

\begin{itemize}
\item Universal, cualquiera puede usarlo.
\item Satisface unas necesidades preestablecidas que pueden satisfacer un 80 de la demanda de peticiones.
\item Software y Hardware provisto por la empresa.
\item El uso del Big Data hace que su rango de aplicaciones sea grande.
\end{itemize}


\item OPORTUNIDADES

\begin{itemize}
\item Mercado del IoT aún por explotar.
\item Empresa joven y dinámica.
\end{itemize}


\item AMENAZAS
\begin{itemize}
\item Posibles competidores con productos similares:
\begin{itemize}
\item WhatsBee de BMates.
\item Family Safe de Intener of Things & Innovation.
\end{itemize}


\item Redes de comunicación aún en desarrollo: NB-IoT y LTE/M.
\item Altos costea de chips NB-IoT y LTE/M.
\item Temas relativos a la Privacidad y Seguridad
\end{itemize}
\end{enumerate}
