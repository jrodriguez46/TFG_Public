\chapter{Conclusiones}
\label{ch:conclusiones}

Las conclusiones deben cerrar el documento, destacando los aspectos más importantes de la ejecución del TFG.  Debe analizar qué objetivos se han alcanzado y en qué grado, qué objetivos se han tenido que dejar fuera del proyecto y por qué, y qué líneas de trabajo futuro abre el TFG.  Para esta última parte es especialmente útil la lista \emph{Blocked} y la lista \emph{Backlog} de tu tablero Trello.

En total las conclusiones no deberían superar dos o tres hojas.  Si ves que se alarga demasiado, traslada material al capítulo de discusión de resultados.

Fíjate en que los objetivos abren el trabajo personal y las conclusiones lo cierran.  Procura mantener un orden que resalte esta relación.