\section{Analisis del ecosistema Universall®}

A continuación se detallan los análisis técnico, funcional y socioeconómicos que se han llevado a cabo en el dispositivo que vamos a emplar para desarrollar la propuesta.

Cabe destacar que el ecosistema Universall® está compuesto por hardware, software y casos de uso.

\subsection{Analisis Técnico de Universall}

En este capítulo se va a llevar  a cabo un analisis técnico del ecosistema Universall, el cual está compuesto tanto por una parte software como una hardware. La datos e información mostradas en este capítulo definen a Universall en términos generales, variando la composición de las partes en función de su aplicación final. Para nuestra apliación concreta se destinará un capítulo exclusivo definiendo el dispositivo que empleramos.

\begin{enumerate}
\item \textbf{HARDWARE}
\end{enumerate}

Universall es un dispositivo IoT que está compuesto por dos partes claramente diferenciadas: los quarks y omega.

\textbf{QUARKS}
Esta parte es la encargada de recopilar toda la información de cada uno de los sensores que componen el quark para más tarde enviarla a la nube para su análisis. En los quarks diferenciamos las siguientes caracterísitcas:
\begin{itemize}
\item Bluetooh 5.0: necesario para la comunicación entre los quarks y omega y entre los propios quarks.
\item Sensor 9D: hace referencia al conjunto de acelerómetro, giroscopio y brújula.
\begin{itemize}
\item Acelerómetro: mide aceleraciones y vibraciones, lo que hace posible saber si el dispositivo se está moviendo.
\item Giroscopio: mide la velicidad angular y detecta cuando el dispositivo está girando. 
\item Brújula: gracias a ella conoceremos la orientación en la que se encuentra nuestro dispositivo.
\end{itemize}
\item Sensor efecto Hall: es un sensor encargado de medir campos mágneticos y cuya función reside
\item Led multicolor: es programable en función del uso final que se dará al dispositivo. Gracias a su gama de colores pueden indicarse diferentes aspectos como si el dipositivo presenta batería baja, se está emparejando con los quarks o por ejemplo si está encendido.
\item Botón de presión: también es programable en función del uso final del producto y puede ser usado para mandar una alarma cuando se pulsa, para emparejarlos o para encenderlos.
\item Batería: puede ser de dos tipos:
\begin{itemize}
\item Pila: opción con una batería de botón intercambiable de 1000mAh cuya vida varía entre los 24 y 36 meses.
\item Batería recargable: opción que cuenta con una batería recargable de 80mAh con una vida aproximada de 3 años, que puede variar en función del uso del quark. 
\end{itemize}
\item Las dimensiones del quark son muy reducidas (30 x 30 x 10 mm).
\end{itemize}

\textbf{OMEGA}
Está parte es la encargada de recoger la información recopilada por los quraks y enviarla a la nube para que el usuario sea capaz de gestionar y analizar la información recogida. Omega está formado por:

\begin{itemize}
\item Wifi y/o NB-IoT: son las encargadas de enviar la información recogida a la nube a los servidores. El uso de una u otra dependerá de la aplicación final que tiene Universall.
\item Bluetooh 5.0:  necesario para la comunicación entre los quarks y omega
\item Batería no recargable: se trata de una batería de SoCl2 destinada a sistemas con una larga esperanza de vida. La estimación de vida de esta batería dependerá del número de conexiones que se realicen y de la cantidad de información subida, aunque se estima en 5 años.
\item USB-C: destinado a la conexión de omega con la red para no agotar la batería no recargable. 
\item Led multicolor: es programable en función del uso final que se dará al dispositivo. Gracias a su gama de colores pueden indicarse diferentes aspectos como si el dipositivo presenta batería baja, se está emparejando con los quarks o por ejemplo si está encendido.
\item Botón de presión: también es programable en función del uso final del producto y puede ser usado para mandar una alarma cuando se pulsa, para emparejarlos o para encenderlos.
\item Sensor efecto Hall: es un sensor encargado de 
\item Elementos opcionales:
\begin{itemize}
\item Sensor 9D
\item Sensor de temperatura
\item Sensor de humedad
\item Sensor de posición
\item Sensor de luz
\item Sensor de calidad del aire
\end{itemize}
\item Tiene unas dimensiones muy reducidas (115mm x 78mm x 35mm) 
\end{itemize}

\textbf{SOFTWARE}



